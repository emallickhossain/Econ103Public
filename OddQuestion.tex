\documentclass[11pt]{article}
\usepackage{enumerate}
\usepackage[margin=1.2in]{geometry}

\title{Seven ``Odd Questions''}
\author{From \emph{An Introduction to Probability and Inductive Logic}\\ by Ian Hacking}
\date{}

\begin{document} 
\maketitle

\noindent Try your luck at these questions, without any calculating.
Each question will be discussed in class.
Do not be surprised if you make mistakes!

\begin{enumerate}
	\item About as many boys as girls are born in hospitals. Many babies are born every week at City 	General. In Cornwall, a country town, there is a small hospital where only a few babies are born 		every week. A \emph{normal} week is one where between 45\% and 55\% of babies are 				females. An \emph{unusual} week is one where more than 55\% are girls, or more than 55\% are 	boys. Which of the following is true:
	\begin{enumerate}[(a)]
		\item Unusual weeks occur equally often at City General and at Cornwall.
		\item Unusual weeks are more common at City General.
		\item Unusual weeks are more common at Cornwall.
	\end{enumerate}
	
	\item Pia is thirty-one years old, single, outspoken, and smart. She was a philosophy major. 			When a student, she was an ardent supporter of Native American rights, and she picketed a 			department store that had no facilities for nursing mothers. 
	Rank the following statements in order from most probable to least probable.
	\begin{enumerate}[(a)]
		\item Pia is an active feminist.
		\item Pia is a bank teller.
		\item Pia works in a small bookstore.
		\item Pia is a bank teller and an active feminist.
		\item Pia is a bank teller and an active feminist who takes yoga classes.
		\item Pia works in a small bookstore and is an active feminist who takes yoga classes.
	\end{enumerate}
	
  	\item In Lotto 6/49, a standard government-run lottery, you choose 6 out of 49 numbers (1 			through 49). You win the biggest prize--maybe millions of dollars--if these 6 are drawn. (The 			prize money is divided between all those who choose the lucky numbers. If no one wins, then 		most of the prize money is put back into next weeks lottery.)
	Suppose your aunt offers you, \emph{free}, a choice between two ticket in the lottery, with 			numbers as shown:
	\begin{enumerate}[A.]
		\item You win if 1, 2, 3, 4, 5, and 6 are drawn.
		\item You win if 39, 36, 32, 21, 14, and 3 are drawn.
	\end{enumerate}
	Do you prefer I, II, or are you indifferent between the two?
	
	\item To throw a total of 7 with a pair of dice, you have to get a 1 and a 6, or a 2 and a 5, or a 3 		and a 4. To throw a total of 6 with a pair of dice, you have to get a 1 and a 5, or a 2 and a 4, or a 3 		and another 3. With two fair dice, you would expect:
		\begin{enumerate}[(a)]
			\item To throw 7 more frequently than 6.
			\item To throw six more frequently than 7.
			\item To throw 6 and 7 equally often.
		\end{enumerate}
		
	\item You have been called to jury duty in a town where there are two taxi companies, Green 			Cab Ltd.\ and Blue Taxi Inc. Blue taxi uses cars painted blue; Green Cabs uses green cars. Green 		cabs dominates the market, with 85\% of the taxis on the road. On a misty winter night a taxi 		sideswiped another car and drove off. A witness says it was a blue cab. The witness is tested 			under conditions like those on the night of the accident, and 80\% of the time she correctly 			reports the color of the cab that is seen. That is, regardless of whether she is shown a blue or a 		green cab in misty evening light, she gets the color right 80\% of the time. You conclude, on the 		basis of this information:
	\begin{enumerate}[(a)]
		\item The probability that the sideswiper was blue is 0.8.
		\item It is more likely that the sideswiper was blue, but the probability is less than 0.8. 
		\item It is just as probable that the sideswiper was green as that it was blue. 
		\item It is more likely than not that the sideswiper was green.
	\end{enumerate}
	
	\item You are a physician. You think it is quite likely that one of your patients has strep throat, 		but you aren't sure. You take some swabs from the throat and send them to a lab for testing. 			The test is (like nearly all lab tests) not perfect.  If the patient has strep throat, then 70\% if the 		time the lab says yes. But 30\% of the time it says NO. If the patient does not have strep throat, 		then 90\% of the time the lab says NO. But 10\% of the time it says YES. You send five 				successive swabs to the lab, from the same patient. and get back these results in order: YES, NO, 	YES, NO, YES. You conclude:
	\begin{enumerate}[(a)]
		\item These results are worthless.
		\item It is likely that the patient does not have strep throat.
		\item It is slightly more likely than not, that the patient does have strep throat.
		\item It is very much more likely than not, that the patient does have strep throat.
	\end{enumerate}
\end{enumerate}
	
\end{document}