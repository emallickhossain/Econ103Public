\documentclass[10pt,letterpaper]{article}
\usepackage{amsmath}
\usepackage{amsfonts}
\usepackage{amssymb}
\usepackage{graphicx}
\usepackage{hyperref}

\title{R Project}
\author{Econ 103}
\date{}
\begin{document}
\maketitle

\section{Description}
The goal of this R project is to give you an opportunity to apply the skills you are learning in class to a question or idea that you find interesting. By the end of the semester, you should be pleasantly surprised at the vast array of questions that you will be able to answer because you have the necessary theoretical knowledge as well as the requisite skills to translate that question (and its answer) into something actionable. \textbf{This could give you a head-start on your honors thesis.}

\section{Can We Work In Groups?}
Yes! Groups (2-3 people) are encouraged. Larger groups would be permitted, but would have to ask me for approval. My expectations for projects will scale with the size of the group because I expect everyone to contribute. 

\section{What Do We Have to Do?}
I want this to be a fun and engaging project. The best way to do that is to have you be working on something that you actually care about or are interested in. I am not placing any kinds of restrictions on topics or subjects. In short, \textbf{as long as we can find data about it, you can do a project on it. } Please discuss with me beforehand though. 

\begin{enumerate}
	\item Find a question that you are interested in answering or exploring
	\item Find a dataset that could be analyzed to answer this question (feel free to discuss with me if you're not sure if or where the data might be)
	\begin{itemize}
		\item FRED, OECD, and the IMF are good sources of macroeconomic data
		\item Google trends has information on keyword popularity
		\item Nate Silver's FiveThirtyEight blog has created some nice data sets on a wide range of subjects
		\item R has packages for directly pulling Twitter data (twitteR) or financial data (Quandl or quantmod)
	\end{itemize}
	\item Summarize, visualize, and/or test hypotheses with your data set
\end{enumerate}

\section{What Do I Turn In?}
I am looking for a short report (ideally done in RMarkdown or \TeX) that includes the following:
\begin{enumerate}
	\item The question being explored
	\item A summary of the data being used and an explanation of why this data is relevant
	\item Tables and charts of relevant summary statistics about the data
	\item Hypothesis tests and/or data visualization
	\item Discussion of testing results
	\item Criticism of your findings. What are it's biggest flaws?
	\item Suggestions for future extensions of this project
\end{enumerate}

\section{What Are Some Examples?}
The R Tutorials provide some examples of the kind of data exploration I am looking for. Other good examples would be the following:
\begin{itemize}
	\item Monthly Council of Economic Advisers (CEA) blog posts on \href{https://www.whitehouse.gov/blog/2016/09/02/employment-situation-august}{jobs} and \href{https://www.whitehouse.gov/blog/2016/08/26/second-estimate-gross-domestic-product-second-quarter-2016}{GDP} 
	\item FiveThirtyEight's \href{http://fivethirtyeight.com/features/gun-deaths/}{summary of gun deaths} (the stellar visualization would be a challenge, but the summary stats are something you can do) 
\end{itemize}

\section{Do I Need To Discover Something New?}
I'm not looking for a Nobel-prize-winning discovery (or even a thesis-level discovery). I want you to learn something you did not know and get experience answering your own questions and developing an outlet for your intellectual curiosity. 

\section{What is the Timeline?}
\begin{itemize}
\item \textbf{October 12:} Submit project idea and question or spoken with me about the project. Submit names of people in your group.

\item \textbf{November 9:} (Optional) Submit summary statistics of your data

\item \textbf{November 16:} (Optional) Submit rough draft for comments.

\item \textbf{December 7:} Hand in project.
\end{itemize}
\end{document}