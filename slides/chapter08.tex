%\documentclass[handout]{beamer}
\documentclass{beamer}
 
\usetheme[numbering = fraction, progressbar = none, background = light, sectionpage = progressbar]{metropolis}
\usepackage{amsmath}
\usepackage{tabu}
\usepackage{graphicx}
\usepackage{hyperref}
\usepackage{xcolor}
\usepackage{tikz}
\usepackage{setspace}
\usetikzlibrary{shapes,backgrounds,trees}

\title{Econ 103 -- Statistics for Economists}
\subtitle{Chapter 8: Hypothesis Testing}
\author{Mallick Hossain}
\date{}
\institute{University of Pennsylvania}
\begin{document} 

%%%%%%%%%%%%%%%%%%%%%%%%%%%%%%%%%%%%%%%%
\begin{frame}
	\titlepage 
\end{frame} 

%%%%%%%%%%%%%%%%%%%%%%%%%%%%%%%%%%%%%%%%
\begin{frame}
\begin{center}
\Huge Hypothesis Testing I
\end{center}
\end{frame}

%%%%%%%%%%%%%%%%%%%%%%%%%%%%%%%%%%%%%%%%
\begin{frame}
\frametitle{An excerpt from \emph{The Lady Tasting Tea} by David Salsburg}
\footnotesize
\begin{quote}
It was a summer afternoon in Cambridge, England, in the late 1920s. A group of university dons, their wives, and some guests were sitting around an outdoor table for afternoon tea. One of the women was insisting that tea tasted different depending upon whether the tea was poured into the milk or whether the milk was poured into the tea. The scientific minds among the men scoffed at this as sheer nonsense. What could be the difference? They could not conceive of any difference in the chemistry of the mixtures that could exist. A thin, short man, with thick glasses and a Vandyke beard beginning to turn gray, pounced on the problem. ``Let us test the proposition'' he said excitedly. He began to outline an experiment in which the lady who insisted there was a diference would be presented with a sequence of cups of tea, in some of which the milk had been poured into the tea and in others of which the tea had been poured into the milk.
\end{quote}
\end{frame}
%%%%%%%%%%%%%%%%%%%%%%%%%%%%%%%%%%%%%%%%
% \begin{frame}

% \begin{figure}
% \includegraphics[scale = 0.35]{./images/grant1}

% \caption{The Orchard, Grantchester}
% \end{figure}

% \end{frame}

% %%%%%%%%%%%%%%%%%%%%%%%%%%%%%%%%%%%%%%%%
% \begin{frame}

% \begin{figure}
% \includegraphics[scale = 0.15]{./images/grant2}

% \caption{What to have with your tea.}
% \end{figure}

% \end{frame}

% %%%%%%%%%%%%%%%%%%%%%%%%%%%%%%%%%%%%%%%%

% \begin{frame}

% \begin{figure}
% \includegraphics[scale = 0.31]{./images/grant6}\\
% \vspace{0.6em}
% \includegraphics[scale = 0.211]{./images/grant7}

% \caption{Why walk when you can punt?}
% \end{figure}

% \end{frame}

% %%%%%%%%%%%%%%%%%%%%%%%%%%%%%%%%%%%%%%%%
% \begin{frame}

% \begin{figure}
% \includegraphics[scale = 0.3]{./images/grant8}

% \caption{What to wear. (Yes, I do in fact own such a hat.)}
% \end{figure}

% \end{frame}

% %%%%%%%%%%%%%%%%%%%%%%%%%%%%%%%%%%%%%%%%
\begin{frame}
\frametitle{Continued...}
\footnotesize
\begin{quote}
And so it was that summer afternoon in Cambridge. The man with the Vandyke beard was Ronald Aylmer Fisher, who was in his late thirties at the time. He would later be knighted Sir Ronald Fisher. In 1935, he wrote a book entitled The Design of Experiments, and he described the experiment of the lady tasting tea in the second chapter of that book. In his book, Fisher discusses the lady and her belief as a hypothetical problem. He considers the various ways in which an experiment might be designed to determine if she could tell the difference.
\end{quote}
\end{frame}

%%%%%%%%%%%%%%%%%%%%%%%%%%%%%%%%%%%%%%%%
\begin{frame}
\begin{center}
\Huge The Pepsi Challenge \\
	\large (Volunteers: 1 ``Expert,'' 1 Skeptic)
\end{center}
\end{frame}
%%%%%%%%%%%%%%%%%%%%%%%%%%%%%%%%%%%%%%%%
\begin{frame}
\frametitle{The Pepsi Challenge}
Our expert claims to be able to tell the difference between Coke and Pepsi. Let's put this to the test! 
\begin{itemize}
\item Eight cups of soda 
	\begin{itemize}
\item Four contain Coke 
\item Four contain Pepsi 
\end{itemize}
	\item The cups are randomly arranged 
	\item How can we use this experiment to tell if our expert can \emph{\alert{really}} tell the difference?
\end{itemize}
\end{frame}
%%%%%%%%%%%%%%%%%%%%%%%%%%%%%%%%%%%%%%%%
\begin{frame}
\frametitle{The Results:}
	\# of Cokes Correctly Identified: \\ \vspace{2em}
	\alert{What do you think? Can our expert really tell the difference? }
		\begin{enumerate}[(a)]
\item Yes
\item No
\end{enumerate}
\end{frame}

%%%%%%%%%%%%%%%%%%%%%%%%%%%%%%%%%%%%%%%%
\begin{frame}
If you just guess randomly, what is the probability of identifying \emph{all four cups of Coke correctly}?
\pause
\begin{itemize}
\item ${8\choose 4}=70$ ways to choose four of the eight cups. \pause
\item If guessing randomly, each of these is \emph{\alert{equally likely}} \pause
\item Only \emph{\alert{one}} of the 70 possibilities corresponds to correctly \pause identifying all four cups of Coke. \pause
\item Thus, the probability is $1/70 \approx 0.014$
\end{itemize}
\end{frame}
%%%%%%%%%%%%%%%%%%%%%%%%%%%%%%%%%%%%%%%%
\begin{frame}
If you just guess randomly, what is the probability of identifying \emph{all but one cup of Coke correctly}?
\pause
\begin{itemize}
\item ${8\choose 4}=70$ ways to choose four of the eight cups. \pause
\item If guessing randomly, each of these is \emph{\alert{equally likely}} \pause
\item There are 16 ways to mis-identify one Coke: 
	\begin{itemize}
		\item $4$ choices of \emph{which} Coke you call a Pepsi 
		\item $4$ choices of \emph{which} Pepsi you call a Coke 
		\item Total of $4\times 4 = 16$ possibilities \pause
	\end{itemize}	
\item Thus, the probability is $16/70 \approx 0.23$
\end{itemize}
\end{frame}
%%%%%%%%%%%%%%%%%%%%%%%%%%%%%%%%%%%%%%%%
\begin{frame}
\frametitle{Probabilities if Guessing Randomly}
	\begin{center}
		\begin{tabular}{rccccc}
		\hline
		\# Correct & 0 & 1 & 2 & 3 & 4\\
		Prob.&1/70 & 16/70 & 36/70 & 16/70 &1/70\\
		\hline
		\end{tabular}
	\end{center}
\end{frame}
%%%%%%%%%%%%%%%%%%%%%%%%%%%%%%%%%%%%%%%%
\begin{frame}
	\begin{center}
		\begin{tabular}{rccccc}
		\hline
		\# Correct & 0 & 1 & 2 & 3 & 4\\
		Prob.&1/70 & 16/70 & 36/70 & 16/70 &1/70\\
		\hline
		\end{tabular}
	\end{center}
	If you're just guessing, what is the probability of identifying \alert{\emph{at least}} three Cokes correctly?
	\pause
	\begin{itemize}
\item Probabilities of mutually exclusive events sum. 
\item $P$(all four correct) = 1/70 
\item $P$(exactly 3 correct )= 16/70 
\item $P$(at least three correct) $ = 17/70 \approx 0.24$

\end{itemize}
\end{frame}

%%%%%%%%%%%%%%%%%%%%%%%%%%%%%%%%%%%%%%%%
\begin{frame}
\frametitle{The Pepsi Challenge}
	\begin{itemize}
\item Even if you're just guessing randomly, the probability of correctly identifying three or more Cokes is around 24\% 
\item In contrast, the probability of identifying \emph{\alert{all four}} Cokes correctly is only around 1.4\% if you're guessing randomly. 
\item We should probably require the expert to get them all right. 
\item What if the expert gets them all wrong? This also has probability $1.4\%$ if you're guessing randomly...
\end{itemize}
\end{frame}

%%%%%%%%%%%%%%%%%%%%%%%%%%%%%%%%%%%%%%%%

\begin{frame}
	\begin{center}
	\huge That was a Hypothesis Test!\\
	\normalsize We'll go through the details in a moment, but first an analogy...
	\end{center}
\end{frame}

%%%%%%%%%%%%%%%%%%%%%%%%%%%%%%%%%%%%%%%%



\begin{frame}
	\begin{center}
	\huge Hypothesis Testing is Similar to a Criminal Trial
	\end{center}
\end{frame}

%%%%%%%%%%%%%%%%%%%%%%%%%%%%%%%%%%%%%%%%

\begin{frame}
%\frametitle{Hypothesis Testing -- Analogy to Criminal Trial}
\footnotesize
\begin{columns}
\begin{column}{6cm} 
 %FIRST COLUMN HERE
   	\begin{block}{Criminal Trial}
	\begin{itemize}
		\item<1-> The person on trial is either innocent or guilty (but not both!)
		\item<2-> ``Innocent Until Proven Guilty''
		\item<3-> Only convict if evidence is ``beyond a shadow of a doubt''
		\item<4-> \emph{Not Guilty} rather than Innocent
			\begin{itemize}\footnotesize
				\item<5-> Acquit $\neq$ Innocent
			\end{itemize}
		\item<6-> Two Kinds of Errors:
			\begin{itemize} \footnotesize
				\item<6-> Convict the innocent
				\item<6->  Acquit the guilty
			\end{itemize}
		\item<7-> Convicting the innocent is a worse error. Want this to be rare even if it means acquitting the guilty.	\end{itemize}
\end{block}
   
   
\end{column} 
\begin{column}{6cm} 

 %SECOND COLUMN HERE 

\begin{block}{Hypothesis Testing}
		\begin{itemize}
		\item<1-> Either the null hypothesis $H_0$ or the alternative $H_1$  hypothesis is true.
		\item<2-> Assume $H_0$ to start
		\item<3-> Only reject $H_0$ in favor of $H_1$ if there is strong evidence.
		\item<4-> \emph{Fail to reject} rather than Accept $H_0$	
		\begin{itemize} \footnotesize
			\item<5-> (Fail to reject $H_0) \neq (H_0$ True) 
		\end{itemize}
				\item<6-> Two Kinds of Errors:
			\begin{itemize} \footnotesize
				\item<6-> Reject true $H_0$ (Type I)
				\item<6-> Don't reject false $H_0$ (Type II)
			\end{itemize}
			\item<7-> Type I errors (reject true $H_0$) are worse: make them rare even if that means more Type II errors.
	\end{itemize}
\end{block}

\end{column} 
\end{columns} 

\end{frame}
%%%%%%%%%%%%%%%%%%%%%%%%%%%%%%%%%%%%%%%%
\begin{frame}
\frametitle{How is the Pepsi Challenge a Hypothesis Test?}
	\begin{block}{Null Hypothesis $H_0$}
		Can't tell the difference between Coke and Pepsi: just guessing. \pause
\end{block}
	\begin{block}{Alternative Hypothesis $H_1$}
	Able to distinguish Coke from Pepsi.\pause
\end{block}
	\begin{block}{Type I Error -- Reject $H_0$ even though it's true} 
	Decide expert can tell the difference when she's really just guessing. \pause
\end{block}
	\begin{block}{Type II Error -- Fail to reject $H_0$ even though it's false}
	Decide expert just guessing when she really can tell the difference. 
\end{block}
\end{frame}
%%%%%%%%%%%%%%%%%%%%%%%%%%%%%%%%%%%%%%%%
\begin{frame}
\frametitle{How do we find evidence to reject $H_0$?}
	\begin{itemize}
		\item Choose a \alert{significance level $\alpha$} maximum probability of Type I error that we are willing to tolerate. 
			\begin{itemize}
				\item Measures how often we will reject a true null, i.e.\ convict an innocent person \pause
			\end{itemize}
		\item Test Statistic $T_n$ uses sample to measure plausibility of $H_0$ \pause
		\item Null Hypothesis $H_0 \Rightarrow$ Sampling Distribution for $T_n$  
			\begin{itemize}
				\item ``Under the null'' means ``assuming the $H_0$ is true'' \pause
			\end{itemize}
		\item Using $\alpha$ and the sampling distribution of $T_n$ under the null, we construct a \alert{decision rule} in terms of a critical value $c_\alpha$ \pause
			\begin{itemize}
				\item Reject $H_0$ if $T_n > c_\alpha$
			\end{itemize}
	\end{itemize}
	
\end{frame}
%%%%%%%%%%%%%%%%%%%%%%%%%%%%%%%%%%%%%%%%
% \begin{frame}
% \frametitle{We still have a random sampling model in mind!}
% \footnotesize
% \begin{block}{Why does $T_n$ have a sampling distribution?}
% 	\begin{itemize}
% 		\item Random Sampling: new data $\Rightarrow$ different \emph{realization} $t$ of $T_n$ 
% 		\item Key point: $T_n$ is a \emph{random variable} with a particular distribution under the null hypothesis $H_0$ \pause
% 	\end{itemize}
% \end{block}

% \begin{block}{What do we mean by $\alpha$?} 
% 	\begin{itemize}
% 		\item $T_n$ is a RV $\Rightarrow$ outcome of hypothesis test is random! 
% 		\item Sometimes we make mistake: either reject $H_0$ when it is true or fail to reject it when it is false. 
% 		\item Repeated Sampling $\Rightarrow$ many different realizations of $T_n\Rightarrow$ many different outcomes of the test. 
% 		\item Test is constructed so that, if $H_0$ is true, we will reject it no more than $100\times \alpha \%$ of the time under repeated sampling.
% 	\end{itemize}
% \end{block}
% \end{frame}


%%%%%%%%%%%%%%%%%%%%%%%%%%%%%%%%%%%%%%%%
\begin{frame}
\frametitle{Example: Pepsi Challenge}
	\begin{block}{Test Statistic $T_n$}
		$T_n =$ Number of Cokes correctly identified
\end{block} 
	\begin{block}{$H_0\colon$ No skill, just guessing randomly}
	Under this null hypothesis, the sampling distribution of $T_n$ is:
		\begin{center}
		\begin{tabular}{rccccc}
		\hline
		\# Correct & 0 & 1 & 2 & 3 & 4\\
		Prob.&1/70 & 16/70 & 36/70 & 16/70 &1/70\\
		\hline
		\end{tabular}
	\end{center}
\end{block}


\end{frame}
%%%%%%%%%%%%%%%%%%%%%%%%%%%%%%%%%%%%%%%%
\begin{frame}
\frametitle{Example: Pepsi Challenge }
$T_n\colon$ \# of Cokes correctly identified. Sampling Dist.\ under $H_0$:
		\begin{center}
		\begin{tabular}{rccccc}
		\hline
		\# Correct & 0 & 1 & 2 & 3 & 4\\
		Prob.&1/70 & 16/70 & 36/70 & 16/70 &1/70\\
		\hline
		\end{tabular}
	\end{center}
	\alert{If I choose a significance level of $\alpha =0.05$, what critical value should I use?}\\ (Remember that $\alpha$ is the probability of rejecting $H_0$ when it is actually true.)
	\pause
	
	\vspace{2em}
	Want $P(\mbox{Reject } H_0|H_0 \mbox{ True})\leq 0.05$\\ 
	$P(T_n \geq 3 |\mbox{Just Guessing}) = 17/70 \approx 0.23 > 0.05$ 
	$P(T_n \geq 4 |\mbox{Just Guessing}) = 1/70 \approx 0.014 \alert{\leq 0.05}$ 
\end{frame}
%%%%%%%%%%%%%%%%%%%%%%%%%%%%%%%%%%%%%%%%
\begin{frame}
\frametitle{Example: Pepsi Challenge }
$T_n\colon$ \# of Cokes correctly identified. Sampling Dist.\ under $H_0$:
		\begin{center}
		\begin{tabular}{rccccc}
		\hline
		\# Correct & 0 & 1 & 2 & 3 & 4\\
		Prob.&1/70 & 16/70 & 36/70 & 16/70 &1/70\\
		\hline
		\end{tabular}
	\end{center}
	\alert{If I choose a significance level of $\alpha =0.25$, what critical value should I use?}
	\pause
	
	\vspace{2em}
	Want $P(\mbox{Reject } H_0|H_0 \mbox{ True})\leq 0.25$\\ 
	$P(T_n \geq 2 |\mbox{Just Guessing}) = 53/70 \approx 0.76 > 0.25$ 
	$P(T_n \geq 3 |\mbox{Just Guessing}) = 17/70 \approx 0.23 \alert{\leq 0.25}$ 
\end{frame}
%%%%%%%%%%%%%%%%%%%%%%%%%%%%%%%%%%%%%%%%



%%%%%%%%%%%%%%%%%%%%%%%%%%%%%%%%%%%%%%%%
\begin{frame}
\frametitle{Example: Pepsi Challenge }
\footnotesize 
$H_0\colon$ Expert is just guessing randomly.\\
$H_1\colon$ Expert can distinguish Coke from Pepsi.\\
$T_n\colon$ \# of Cokes correctly identified. Has following sampling under the null:
		\begin{center}
		\begin{tabular}{rccccc}
		\hline \footnotesize
		\# Correct & 0 & 1 & 2 & 3 & 4\\
		Prob.&1/70 & 16/70 & 36/70 & 16/70 &1/70\\
		\hline
		\end{tabular}
	\end{center}
	\vspace{2em}
	\normalsize
	\alert{If I choose $\alpha =0.05$, what decision rule should I use?}
	\begin{enumerate}[(a)]
		\item Reject $H_0$ if $T_n \geq 0$
		\item Reject $H_0$ if $T_n \geq 1$
		\item Reject $H_0$ if $T_n \geq 2$
		\item Reject $H_0$ if $T_n \geq 3$
		\item Reject $H_0$ if $T_n \geq 4$
	\end{enumerate}
\end{frame}
%%%%%%%%%%%%%%%%%%%%%%%%%%%%%%%%%%%%%%%%
\begin{frame}
\frametitle{Example: Pepsi Challenge}
\footnotesize 
$H_0\colon$ Expert is just guessing randomly.\\
$H_1\colon$ Expert can distinguish Coke from Pepsi.\\
$T_n\colon$ \# of Cokes correctly identified. Has following sampling under the null:
		\begin{center}
		\begin{tabular}{rccccc}
		\hline \footnotesize
		\# Correct & 0 & 1 & 2 & 3 & 4\\
		Prob.&1/70 & 16/70 & 36/70 & 16/70 &1/70\\
		\hline
		\end{tabular}
	\end{center}
	\vspace{2em}
	\normalsize
	\alert{If I choose $\alpha =0.05$, what decision rule should I use?}\\
\vspace{1em}
	Need $P(\mbox{Reject } H_0|H_0 \mbox{ True})\leq \alpha = 0.05$ 
	\begin{eqnarray*}
		P(T_n \geq 3 |\mbox{Just Guessing}) &=& 17/70 \approx 0.23 > 0.05 \\
	P(T_n \geq 4 |\mbox{Just Guessing}) &=& 1/70  \approx 0.014 \alert{\leq 0.05}
	\end{eqnarray*}
	\vspace{1em}
	\alert{Critical value for $\alpha = 0.05$ is 4}
\end{frame}
%%%%%%%%%%%%%%%%%%%%%%%%%%%%%%%%%%%%%%%%

\begin{frame}
\frametitle{Example: Pepsi Challenge }
\footnotesize 
$H_0\colon$ Expert is just guessing randomly.\\
$H_1\colon$ Expert can distinguish Coke from Pepsi.\\
$T_n\colon$ \# of Cokes correctly identified. Has following sampling under the null:
		\begin{center}
		\begin{tabular}{rccccc}
		\hline \footnotesize
		\# Correct & 0 & 1 & 2 & 3 & 4\\
		Prob.&1/70 & 16/70 & 36/70 & 16/70 &1/70\\
		\hline
		\end{tabular}
	\end{center}
	\vspace{2em}
	\normalsize
	\alert{If I choose $\alpha =0.25$, what critical value should I use?}
	\begin{enumerate}[(a)]
		\item 0
		\item 1
		\item 2
		\item 3
		\item 4
	\end{enumerate}
\end{frame}
%%%%%%%%%%%%%%%%%%%%%%%%%%%%%%%%%%%%%%%%
%\begin{frame}
%At what significance levels would we reject the assumption that our expert was guessing randomly?
%\end{frame}
%%%%%%%%%%%%%%%%%%%%%%%%%%%%%%%%%%%%%%%%%
%
%\begin{frame}
%\frametitle{P-values}
%Reject or fail to reject at a given significance level is a somewhat crude: doesn't tell us whether it was ``close'' or ``no contest.'' P-value is more informative: \pause
%	\begin{block}{P-Value = Strength of evidence against the null}
%	The p-value for a hypothesis test is the probability that we would observe a test statistic \emph{at least as extreme} as the one actually observed \emph{if the null hypothesis were true}.
%\end{block}
%\pause
%\vspace{1em}
%\alert{This means that the p-value equals the \emph{smallest significance level} at which our observed test statistic would cause us to reject $H_0$. It may be tempting to conclude that a p-value is the probability that the null is true but this is NOT CORRECT.}
%\end{frame}
%
%\begin{frame}
%\frametitle{Example: Pepsi Challenge}
%\footnotesize 
%$H_0\colon$ Expert is just guessing randomly.\\
%$H_1\colon$ Expert can distinguish Coke from Pepsi.\\
%$T_n\colon$ \# of Cokes correctly identified. Has following sampling under the null:
%		\begin{center}
%		\begin{tabular}{rccccc}
%		\hline \footnotesize
%		\# Correct & 0 & 1 & 2 & 3 & 4\\
%		Prob.&1/70 & 16/70 & 36/70 & 16/70 &1/70\\
%		\hline
%		\end{tabular}
%	\end{center}
%	\vspace{2em}
%	\normalsize
%	\alert{Our observed test statistic was $\boxed{?}$ since the expert correctly identified this many Cokes. What is the p-value in this case?}\pause
%	\begin{eqnarray*}
%		P(T_n \geq \alert{\boxed{?}}|H_0 \mbox{ True}) = 
%	\end{eqnarray*}
%\end{frame}
%
%%%%%%%%%%%%%%%%%%%%%%%%%%%%%%%%%%%%%%%%

%%%%%%%%%%%%%%%%%%%%%%%%%%%%%%%%%%%%%%%%
\begin{frame}
\begin{block}{Last Time}
Simple Example of Hypothesis Testing: the Pepsi Challenge
\end{block}

\begin{block}{Today and Next Two Lectures}
Hypothesis Testing More Generally
\end{block}

\end{frame}
%%%%%%%%%%%%%%%%%%%%%%%%%%%%%%%%%%%%%%%%

\begin{frame}
\frametitle{Hypothesis: Assertion about Population(s)}
	\begin{itemize}
	\item A Big Mac contains, on average, 550 kcal: \alert{$\mu = 550$}
	\item Midterm 2 was harder than Midterm 1: \alert{$\mu_{1} >\mu_2$}
	\item Equal proportions of Republicans and Democrats know that John Roberts is the chief justice of SCOTUS: \alert{$p = q$}
	\item Google stock is riskier than IBM stock: \alert{$\sigma^2_{X} > \sigma^2_{Y}$}
	\item There is no correlation between height and income: \alert{$\rho = 0$} 
	\end{itemize}
\end{frame}

%%%%%%%%%%%%%%%%%%%%%%%%%%%%%%%%%%%%%%%%
\begin{frame}
	\frametitle{Hypothesis Testing: Try to 
	Find Evidence \emph{Against} $H_0$}
\begin{block}
	{Null Hypothesis: $H_0$}
	\begin{itemize}
		\item Start off assuming $H_0$ is true --
		 ``innocent until proven guilty''
		\item ``Under the Null'' = Assuming the null is true
		\item $H_0$ $\Rightarrow$ know something about population, can calculate probs.
	\end{itemize}
\end{block}
\begin{alertblock}
	{This Course: \emph{Simple} Null Hypotheses}
	$H_0\colon f(\mbox{Parameters}) = \mbox{Known Constant}$, for example
	\begin{itemize}
		\item $\mu_1 - \mu_2 = 0$
		\item $p = 0.5$
		\item $\mu = 0$
		\item $\sigma^2_X/\sigma^2_Y = 1$
	\end{itemize}
\end{alertblock}
\end{frame}
%%%%%%%%%%%%%%%%%%%%%%%%%%%%%%%%%%%%%%%%
\begin{frame}
	\frametitle{How do I know what my null hypothesis is?}
	There is no rule I can give you for this: it depends on the problem. Here are some guidelines:
	\begin{itemize}
		\item It will take the form $f(\mbox{Parameters}) = \mbox{Known Constant}$
		\item Nulls are typically things like ``there is no effect,'' ``these two groups are not different,'' i.e.\ the \emph{status quo}. 
		\item Nulls are \emph{very specific}: we need to be able to do probability calculations under the null -- c.f.\ the Pepsi Challenge.
	\end{itemize}
\end{frame}
%%%%%%%%%%%%%%%%%%%%%%%%%%%%%%%%%%%%%%%%
\begin{frame}[t]
	\frametitle{Example: How many calories in a Big Mac? }
\begin{itemize}
	\item According to McDonald's: 550 kcal on average
	\item Measure calories in random sample of $9$ Big Macs: $X_1, \hdots, X_{9} \sim \mbox{iid } N(\mu, \sigma^2)$
\end{itemize}

\vspace{1em}

\alert{If we wanted to test McDonald's claim, what would be $H_0$?}
\begin{enumerate}[(a)]
	\item $\sigma^2 = 1$
	\item $\mu = 0$
	\item $\mu > 550$ 
	\item $\mu = 550$
	\item $\mu \neq 550$
\end{enumerate}
\end{frame}
%%%%%%%%%%%%%%%%%%%%%%%%%%%%%%%%%%%%%%%%
\begin{frame}[t]
	\frametitle{Example: How many calories in a Big Mac? }
\begin{itemize}
	\item According to McDonald's: 550 kcal on average
	\item Measure calories in random sample of $9$ Big Macs: $X_1, \hdots, X_{9} \sim \mbox{iid } N(\mu, \sigma^2)$
\end{itemize}

\vspace{1em}

\alert{If McDonald's is telling the truth, approximately what value should we get for the sample mean caloric content of the $9$ Big Macs?} 
\end{frame}
%%%%%%%%%%%%%%%%%%%%%%%%%%%%%%%%%%%%%%%%
\begin{frame}[t]
	\frametitle{Example: How many calories in a Big Mac? }
\begin{itemize}
	\item According to McDonald's: 550 kcal on average
	\item Measure calories in random sample of $9$ Big Macs: $X_1, \hdots, X_{9} \sim \mbox{iid } N(\mu, \sigma^2)$
\end{itemize}

\vspace{1em}

\alert{If the sample mean does not equal 550, does this prove that McDonald's is lying?}
\begin{enumerate}[(a)]
	\item Yes
	\item No
	\item Not Sure
\end{enumerate}
\end{frame}
%%%%%%%%%%%%%%%%%%%%%%%%%%%%%%%%%%%%%%%%
\begin{frame}
	\frametitle{How to find evidence against $H_0$? Test Statistic!}
	\begin{block}
		{Test Statistic: $T_n$}
		A statistic that gives us information about the parameter we are testing and has a \emph{known} sampling distribution \emph{under $H_0$}.
	\end{block}
\end{frame}
%%%%%%%%%%%%%%%%%%%%%%%%%%%%%%%%%%%%%%%%
%\begin{frame}[t]
%	\frametitle{Example: How many calories in a Big Mac? }
%\begin{itemize}
%	\item Measure calories in random sample of $n$ Big Macs: $X_1, \hdots, X_9 \sim \mbox{iid } N(\mu, \sigma^2)$
%	\item $H_0\colon \mu = 550$
%\end{itemize}
%
%\vspace{1em}
%
%\alert{Which of these should we use as our test statistic?}
%\begin{enumerate}[(a)]
%	\item $S^2$
%	\item $\bar{X} - 550$
%	\item $\bar{X}$
%	\item $\bar{X} / S$ 
%	\item $(\bar{X} - 550)/(S/3)$
%\end{enumerate}
%\end{frame}
%%%%%%%%%%%%%%%%%%%%%%%%%%%%%%%%%%%%%%%%
\begin{frame}[t]
	\frametitle{Example: How many calories in a Big Mac? }
\begin{itemize}
	\item Measure calories in random sample of $n$ Big Macs: $X_1, \hdots, X_9 \sim \mbox{iid } N (\mu, \sigma^2)$
	\item $H_0\colon \mu = 550$
\end{itemize}

\vspace{1em}

\alert{If McDonald's is telling the truth, i.e.\ under the null, what is \emph{exact}  sampling distribution of $(\bar{X} - 550)/(S/3)$?}
\begin{enumerate}[(a)]
	\item $\chi^2_{9}$
	\item $N(550, 1)$
	\item $F(9, 1)$
	\item $N(0,1)$ 
	\item $t_{8}$
\end{enumerate}
\end{frame}
%%%%%%%%%%%%%%%%%%%%%%%%%%%%%%%%%%%%%%%%
\begin{frame}
	\frametitle{What if the null is false?}

	\begin{block}
		{Alternative hypothesis: $H_1$}
		The \emph{negation} of the null hypothesis.
	\end{block}
	\begin{block}
		{Examples:}
		\begin{enumerate}
			\item 
				\begin{itemize}
					\item $H_0\colon$ This parameter equals 5.
				\item $H_1\colon$ This parameter does \emph{not} equal 5.
		\end{itemize}
			\item 
				\begin{itemize}
					\item $H_0\colon$ There is no difference between these two groups.
					\item $H_1\colon$ There \emph{is} a difference between these two groups.
				\end{itemize}
		\end{enumerate}
		
	\end{block}

	\alert{Sometimes we only care about \emph{certain kinds} of violations of $H_0$...}
\end{frame}
%%%%%%%%%%%%%%%%%%%%%%%%%%%%%%%%%%%%%%%%
\begin{frame}
\frametitle{One-sided vs.\ Two-sided Alternative}
\alert{Let $\theta$ be a population parameter and $\theta_0$ be a specified constant.}
\begin{block}
	{Null Hypothesis}
\begin{itemize}
	\item $H_0\colon \theta = \theta_0$
\end{itemize}\end{block}
	\begin{block}{Two-sided Alternative}
		\begin{itemize}
			\item $H_1\colon \theta \neq \theta_0$
		\end{itemize}
\end{block}
	\begin{block}{One-sided Alternative}
		Two possibilities, depending on the problem at hand:
		\begin{itemize}
			\item $H_1\colon \theta > \theta_0$
			\item $H_1\colon \theta < \theta_0$
		\end{itemize}
\end{block}
\end{frame}

%%%%%%%%%%%%%%%%%%%%%%%%%%%%%%%%%%%%%%%%
\begin{frame}
\frametitle{Example: Suing McDonald's }

A class action lawsuit claims that McDonald's has been  understating the caloric content of the ``Big Mac,'' misleading consumers into thinking the sandwich is healthier than it really is. McDonald's claims the sandwich contains $550$ kcal on average. \\

\vspace{1em}
\alert{Suppose you're the judge in this case. What is your alternative hypothesis?}

	\begin{enumerate}[(a)]
		\item $H_1\colon \mu \neq 550$ kcal
		\item $H_1\colon \mu < 550$ kcal
		\item $H_1\colon \mu > 550$ kcal
		\item $H_1\colon \mu = 550$ kcal
\end{enumerate}
\end{frame}
%%%%%%%%%%%%%%%%%%%%%%%%%%%%%%%%%%%%%%%%

\begin{frame}
\frametitle{Example: Quality Control at McDonald's }

You are a senior manager at McDonald's and are concerned that franchises may be deviating from company policy on the calorie count of a Big Mac sandwich, which is supposed to be 550 kcal on average. Because intervening is costly, you will only take action is there is strong evidence of deviation from company policy. \\

\vspace{1em}

\alert{What is your alternative hypothesis?}
	\begin{enumerate}[(a)]
		\item $H_1\colon \mu \neq 550$ kcal
		\item $H_1\colon \mu < 550$ kcal
		\item $H_1\colon \mu > 550$ kcal
		\item $H_1\colon \mu = 550$ kcal
\end{enumerate}
\end{frame}
%%%%%%%%%%%%%%%%%%%%%%%%%%%%%%%%%%%%%%%%%
\begin{frame}
	\frametitle{Decision Rule: When should we reject $H_0$?}
	\begin{itemize}
		\item Test statistic: RV with known sampling distribution under $H_0$
		\item McDonald's Example: $T_n = 3(\bar{X} - 550)/S$
		\item \emph{Random} since $\bar{X}$ and $S$ are RVs under random sampling: functions of $X_1, \hdots, X_9$.
		\item Observed dataset: \emph{realizations} $x_1, \hdots, x_9$ of RVs $X_1, \hdots, X_9$
		\item Plug in observed data to get estimates (constants) $\bar{x}$ and $s$.
		\item Plug these into the formula for the test statistic to get a \emph{number} -- this is a \emph{realization} of $T_n$ 
		\item Depending on this number, decide whether to reject $H_0$.
	\end{itemize}
\end{frame}
%%%%%%%%%%%%%%%%%%%%%%%%%%%%%%%%%%%%%%%%%
\begin{frame}
\frametitle{What Form Should the Decision Rule Take?}
\begin{columns}
\begin{column}{6cm}
\includegraphics[scale = 0.5]{./images/t_pdf}
\end{column}

\begin{column}{6cm}
$H_0\colon \mu=550 \Rightarrow \displaystyle \frac{\bar{X} - 550}{S/3} \sim t(8)$\\ \pause
\vspace{1em}
One-sided Alternative $H_1\colon \mu > 550$\\ \pause
\vspace{1em}
Two-sided Alternative $H_1\colon \mu \neq 550$ 
\end{column}

\end{columns}
 
\end{frame}
%%%%%%%%%%%%%%%%%%%%%%%%%%%%%%%%%%%%%%%%
\begin{frame}
\frametitle{Example: Suing McDonald's }
The plaintiffs allege that McDonald's has \emph{understated} the true caloric content of a Big Mac: it's actually \emph{greater} than 550 kcal. \alert{Suppose the plaintiffs are right. Then what sort of value should we expect the test statistic $3(\bar{X} - 550)/S$ to take on?}

\vspace{1em}
\begin{enumerate}[(a)]
	\item A value \emph{less} than zero.
	\item A value close to zero.
	\item A value \emph{greater} than zero.
\end{enumerate}
\end{frame}
%%%%%%%%%%%%%%%%%%%%%%%%%%%%%%%%%%%%%%%%
\begin{frame}
\frametitle{Example: Quality Control at McDonald's }
The senior manager is worried that franchises are deviating from company policy that Big Macs should contain approximately 550 kcal. \alert{If the franchises \emph{are} deviating, what sort of of value should we expect the test statistic $3(\bar{X} - 550)/S$ to take on?}

\vspace{1em}
\begin{enumerate}[(a)]
	\item A value \emph{less} than zero.
	\item A value close to zero.
	\item A value \emph{greater} than zero.
	\item A value different from zero but we can't tell whether it will be positive or negative.
\end{enumerate}
\end{frame}
%%%%%%%%%%%%%%%%%%%%%%%%%%%%%%%%%%%%%%%%
\begin{frame}
\frametitle{What Form Should the Decision Rule Take?}
$X_1, \hdots, X_n \sim \mbox{iid } N(\mu, \sigma^2)$ 
\begin{block}{Common Null Hypothesis $H_0\colon \mu = 550$}
Under $H_0$, $T_n = \sqrt{n}(\bar{X}_n - 550)/S \sim t(n-1)$ 
\end{block}
\begin{block}{One-sided Alternative $H_1\colon \mu > 550$}
Reject $H_0$ if $T_n$ is ``too big'' 
\end{block}
\begin{block}{Two-sided Alternative $H_1\colon \mu \neq 550$} 
Reject $H_0$ if $T_n$ is ``too big'' or ``too small''
\end{block}

\vspace{1em}

\alert{But how big of a discrepancy is ``big enough'' to reject?}
\end{frame}

%%%%%%%%%%%%%%%%%%%%%%%%%%%%%%%%%%%%%%%%
\begin{frame}
	\frametitle{Two Kinds of Mistakes in Hypothesis Testing}
	\begin{block}
		{Type I Error}
		\begin{itemize}
			\item Rejecting the null when it's actually true.
			\item $P(\mbox{Type I Error}) = \alpha\quad \quad$ 
			\alert{$\boxed{\alpha= \mbox{``Significance Level'' of Test}}$}
		\end{itemize}
	\end{block}
	 \begin{block}
		{Type II Error}
		\begin{itemize}
			\item Failing to reject the null when it's false.
			\item $P(\mbox{Type II Error}) = \beta \quad \quad$ 
			\alert{$\boxed{1 - \beta= \mbox{``Power'' of Test}}$}
		\end{itemize}
	\end{block}
	\begin{alertblock}
		{Important!}
		Hypothesis testing \emph{controls} probability of a Type I error since this is assumed to be the \emph{worse} kind of mistake: convicting the innocent.	
	\end{alertblock}
\end{frame}
%%%%%%%%%%%%%%%%%%%%%%%%%%%%%%%%%%%%%%%%%
\begin{frame}
	\frametitle{Construct a Decision Rule to \emph{Fix} $\alpha$ at User-Chosen Level}

	\begin{block}
		{Critical Value $c_{\alpha}$} 
	\begin{itemize}
		\item Threshold for rejecting $H_0$
		\item Chosen so that $P(\mbox{Reject } H_0|H_0 \mbox{ is True}) = \alpha$
		\item Depends on \emph{both} $\alpha$ \emph{and} the alternative hypothesis.
	\end{itemize}
	\end{block}
	\begin{block}
		{One-Sided Alternative}
		Reject $H_0$ if $T_n >$ Critical Value
	\end{block}
	\begin{block}
		{Two-Sided Alternative}
		Reject $H_0$ if $|T_n| >$ Critical Value
	\end{block}
\end{frame}
%%%%%%%%%%%%%%%%%%%%%%%%%%%%%%%%%%%%%%%%%
\begin{frame}
\frametitle{Example: One-sided Alternative $H_1\colon \mu > 550$}
The critical value is chosen to reflect both the alternative hypothesis and the significance level. 
\begin{figure}
\includegraphics[scale = 0.45]{./images/one_side}
\end{figure}
One-sided Critical Value: \texttt{qt($1-\alpha$, df  = $n-1$)}
\end{frame}


%%%%%%%%%%%%%%%%%%%%%%%%%%%%%%%%%%%%%%%%

\begin{frame}
\frametitle{Example: Two-sided Alternative $H_1\colon \mu \neq 550$}
The critical value is chosen to reflect both the alternative hypothesis and the significance level. 
\begin{figure}
\includegraphics[scale = 0.45]{./images/two_side}
\end{figure}
Two-sided Critical Value: \texttt{qt($1-\alpha/2$, df  = $n-1$)}
\end{frame}

%%%%%%%%%%%%%%%%%%%%%%%%%%%%%%%%%%%%%%%%
\begin{frame}
Suppose, for example, $\alpha = 0.05$, $n = 9$
	\begin{eqnarray*}
		&&\texttt{qt(0.95, df  = 8)}\approx 1.86\\
		 &&\texttt{qt(0.975, df  = 8)}\approx 2.3
	\end{eqnarray*}
\begin{figure}
\includegraphics[scale = 0.3]{./images/one_side}
\includegraphics[scale = 0.3]{./images/two_side}
\end{figure}
One-sided Alternative: Reject $H_0$ if $3(\bar{X}_n - 550)/S \geq 1.86$\\
\vspace{0.5em}
Two-sided Alternative: Reject $H_0$ if $\left|3(\bar{X}_n - 550)/S\right| \geq 2.3$\\

\end{frame}

%%%%%%%%%%%%%%%%%%%%%%%%%%%%%%%%%%%%%%%%
\begin{frame}
\frametitle{McDonald's Example}
Suppose $n=9$, $\bar{x} = 563$, $s = 34$. What is  the value of our test statistic?

\pause
\vspace{1em}
	$$\frac{563 - 550}{34/\sqrt{9}}= \frac{13}{34/3} \approx 1.14$$


\end{frame}

%%%%%%%%%%%%%%%%%%%%%%%%%%%%%%%%%%%%%%%%
\begin{frame}[t]
\frametitle{McDonald's Example: $\alpha = 0.05$}
Recall that:
\begin{eqnarray*}
		&&\texttt{qt(0.95, df  = 8)}\approx 1.86\\
		 &&\texttt{qt(0.975, df  = 8)}\approx 2.3
	\end{eqnarray*}
Based on an observed test statistic of $1.14$, would we reject $H_0$ against the one-sided alternative at the 5\% significance level?
\begin{enumerate}[(a)]
	\item Yes
	\item No
	\item Not Sure
\end{enumerate}

\end{frame}
%%%%%%%%%%%%%%%%%%%%%%%%%%%%%%%%%%%%%%%%
\begin{frame}[t]
\frametitle{McDonald's Example: $\alpha = 0.05$}
Recall that:
\begin{eqnarray*}
		&&\texttt{qt(0.95, df  = 8)}\approx 1.86\\
		 &&\texttt{qt(0.975, df  = 8)}\approx 2.3
	\end{eqnarray*}
Based on an observed test statistic of $1.14$, would we reject $H_0$ against the \alert{two-sided} alternative at the 5\% significance level?
\begin{enumerate}[(a)]
	\item Yes
	\item No
	\item Not Sure
\end{enumerate}

\end{frame}

%%%%%%%%%%%%%%%%%%%%%%%%%%%%%%%%%%%%%%%%

\begin{frame}
	\frametitle{Reporting the Results of a Hypothesis Test}
	\begin{block}
		{Lawsuit Example}
		The judge \emph{failed to reject} the null hypothesis that $\mu = 550$ against the one-sided alternative $\mu > 550$ at the 5\% significance level.
	\end{block}
	\begin{block}
		{Quality Control Example}
		The senior manager \emph{failed to reject} the null hypothesis that $\mu =550$ against the two-sided alternative at the 5\% significance level.
	\end{block}
	\begin{block}
		{Interpretation}
		In each of these two cases, there was \emph{insufficient evidence} the initial assumption that $\mu = 550$ \emph{given the significance level used}.
	\end{block}
	\alert{But what if we have used a \emph{different} significance level?}
\end{frame}
%%%%%%%%%%%%%%%%%%%%%%%%%%%%%%%%%%%%%%%%
\begin{frame}
	\frametitle{The P-Value of a Hypothesis Test}
	\begin{block}
		{Two Equivalent Definitions:}
		\begin{enumerate}
			\item Given the value we calculated for our test statistic, what is the \emph{smallest $\alpha$} at which we would have rejected the null?
			\item Under the null, what is the probability of observing a test statistic \emph{at least as extreme} as the one we \emph{actually} observed?
		\end{enumerate}
	\end{block}
	\begin{block}
		{Why Report P-Values?}
		\begin{itemize}
			\item More informative than reporting $\alpha$ and Reject/Fail to Reject
			\item E.g. a p-value of 0.03 means we would have rejected the null for any $\alpha \geq 0.03$ and failed to reject it for any $\alpha < 0.03$ 
		\end{itemize}
	\end{block}
\end{frame}
%%%%%%%%%%%%%%%%%%%%%%%%%%%%%%%%%%%%%%%%

\begin{frame}
\begin{center}
\huge P-Value Depends on Which Alternative We Have Specified!
\end{center}
\end{frame}

%%%%%%%%%%%%%%%%%%%%%%%%%%%%%%%%%%%%%%%%
\begin{frame}
\frametitle{What is the p-value? (One-sided Test)}
\footnotesize
Recall: p-value is \emph{smallest significance level} at which our observed test statistic would cause us to reject $H_0$. \alert{Test statistic is $1.14$. What is the one-sided p-value? }
\begin{figure}
\includegraphics[scale= 0.4]{./images/p_upper1}

\end{figure}

\end{frame}

%%%%%%%%%%%%%%%%%%%%%%%%%%%%%%%%%%%%%%%%
\begin{frame}
\frametitle{What is the p-value? (One-sided Test)}
\footnotesize
Recall: p-value is \emph{smallest significance level} at which our observed test statistic would cause us to reject $H_0$. \alert{Test statistic is $1.14$. What is the one-sided p-value? }
\begin{figure}
\includegraphics[scale= 0.4]{./images/p_upper2}

\end{figure}

\end{frame}

%%%%%%%%%%%%%%%%%%%%%%%%%%%%%%%%%%%%%%%%
\begin{frame}
\frametitle{What is the p-value? (One-sided Test)}
\footnotesize
Recall: p-value is \emph{smallest significance level} at which our observed test statistic would cause us to reject $H_0$. \alert{Test statistic is $1.14$. What is the one-sided p-value? }
\begin{figure}
\includegraphics[scale= 0.4]{./images/p_upper3}

\end{figure}

\end{frame}

%%%%%%%%%%%%%%%%%%%%%%%%%%%%%%%%%%%%%%%%
\begin{frame}
\frametitle{What is the p-value? (One-sided Test)}
\footnotesize
Recall: p-value is \emph{smallest significance level} at which our observed test statistic would cause us to reject $H_0$. \alert{Test statistic is $1.14$. What is the one-sided p-value? }
\begin{figure}
\includegraphics[scale= 0.4]{./images/p_upper4}

\end{figure}

\end{frame}

%%%%%%%%%%%%%%%%%%%%%%%%%%%%%%%%%%%%%%%%
\begin{frame}
\frametitle{What is the p-value? (One-sided Test)}
\footnotesize
Recall: p-value is \emph{smallest significance level} at which our observed test statistic would cause us to reject $H_0$. \alert{Test statistic is $1.14$. What is the one-sided p-value? }
\begin{figure}
\includegraphics[scale= 0.4]{./images/p_upper4}

\end{figure}
\texttt{1 - pt(1.14, df = 8)}$\approx 0.14$
\end{frame}

%%%%%%%%%%%%%%%%%%%%%%%%%%%%%%%%%%%%%%%%
\begin{frame}
\frametitle{What is the p-value? (Two-sided Test)}
\footnotesize
Recall: p-value is \emph{smallest significance level} at which our observed test statistic would cause us to reject $H_0$. \alert{Test statistic is $1.14$. What is the two-sided p-value? }
\begin{figure}
\includegraphics[scale= 0.4]{./images/p_both1}

\end{figure}

\end{frame}

%%%%%%%%%%%%%%%%%%%%%%%%%%%%%%%%%%%%%%%%
\begin{frame}
\frametitle{What is the p-value? (Two-sided Test)}
\footnotesize
Recall: p-value is \emph{smallest significance level} at which our observed test statistic would cause us to reject $H_0$. \alert{Test statistic is $1.14$. What is the two-sided p-value? }
\begin{figure}
\includegraphics[scale= 0.4]{./images/p_both2}

\end{figure}

\end{frame}

%%%%%%%%%%%%%%%%%%%%%%%%%%%%%%%%%%%%%%%%
\begin{frame}
\frametitle{What is the p-value? (Two-sided Test)}
\footnotesize
Recall: p-value is \emph{smallest significance level} at which our observed test statistic would cause us to reject $H_0$. \alert{Test statistic is $1.14$. What is the two-sided p-value? }
\begin{figure}
\includegraphics[scale= 0.4]{./images/p_both3}

\end{figure}

\end{frame}

%%%%%%%%%%%%%%%%%%%%%%%%%%%%%%%%%%%%%%%%
\begin{frame}
\frametitle{What is the p-value? (Two-sided Test)}
\footnotesize
Recall: p-value is \emph{smallest significance level} at which our observed test statistic would cause us to reject $H_0$. \alert{Test statistic is $1.14$. What is the two-sided p-value? }
\begin{figure}
\includegraphics[scale= 0.4]{./images/p_both4}

\end{figure}

\end{frame}

%%%%%%%%%%%%%%%%%%%%%%%%%%%%%%%%%%%%%%%%
\begin{frame}
\frametitle{What is the p-value? (Two-sided Test)}
\footnotesize
Recall: p-value is \emph{smallest significance level} at which our observed test statistic would cause us to reject $H_0$. \alert{Test statistic is $1.14$. What is the two-sided p-value? }
\begin{figure}
\includegraphics[scale= 0.4]{./images/p_both5}

\end{figure}

\end{frame}

%%%%%%%%%%%%%%%%%%%%%%%%%%%%%%%%%%%%%%%%
\begin{frame}
\frametitle{What is the p-value? (Two-sided Test)}
\footnotesize
Recall: p-value is \emph{smallest significance level} at which our observed test statistic would cause us to reject $H_0$. \alert{Test statistic is $1.14$. What is the two-sided p-value? }
\begin{figure}
\includegraphics[scale= 0.4]{./images/p_both5}
\end{figure}

\texttt{2 * pt(-1.14, df = 8)}$\approx 0.28$ \pause \hfill \alert{This is twice the one-sided p-value!}
\end{frame}

%%%%%%%%%%%%%%%%%%%%%%%%%%%%%%%%%%%%%%%%

\begin{frame}
\frametitle{Two-sided Test is More Stringent}
\begin{block}{P-value measures strength of evidence against $H_0$}
Lower p-value means stronger evidence. 
\end{block}

\begin{block}{(Two-sided p-value) $= 2 \; \times$  (one-sided p-value)}
Reject $H_0$ based on two-sided test $\implies$ Reject $H_0$ based on appropriate one-sided test. The converse is \emph{false}.
\end{block}


\end{frame}
%%%%%%%%%%%%%%%%%%%%%%%%%%%%%%%%%%%%%%%%

\begin{frame}
\frametitle{Steps in Hypothesis Testing}

\begin{enumerate}
\item Specify Null and Alternative Hypotheses
\item Identify a Test Statistic: a function of the data that has a known sampling distribution under the null.
\item Specify a Decision Rule and a Critical Value so the Type I Error Rate equals $\alpha$.
\end{enumerate}

\begin{alertblock}{Alternative to Step 3}
	Calculate P-Value: the minimum significance level  ($\alpha$) at which we would reject $H_0$ given the observed data.
\end{alertblock}

\end{frame}


%%%%%%%%%%%%%%%%%%%%%%%%%%%%%%%%%%%%%%%%
\begin{frame}
\frametitle{How to Handle Other Examples?}

\alert{You already know lots of sampling distributions! Testing is very similar to constructing confidence intervals in that the steps are always the same, and the only thing that differs is \emph{which} sampling distribution we work with. We'll look at more examples next time.}

\end{frame}

\end{document}