\documentclass[12pt,letterpaper]{article}
\usepackage[utf8]{inputenc}
\usepackage{amsmath}
\usepackage{amsfonts}
\usepackage{amssymb}
\usepackage{graphicx}
\usepackage{enumitem}
\usepackage[margin=1in]{geometry}

\title{How to Do Well in Econ 103}
\author{Adapted from Professor Francis J.\ DiTraglia}
\date{}
\begin{document}

\maketitle

\section{Econ 103 is un-crammable: clock in every week}
Supposedly, you can pass (or even ace) some courses by cramming before each exam. Econ 103 is not that kind of class and that's just a function of the subject matter as opposed to the volume of material. Honestly, if it was crammable, we would have much more sophisticated political discourse since everyone would understand basic statistics. This course is meant to train your brain to think differently, just like learning the violin or training for a marathon. Your body cannot be trained to the requisite proficiency level by spending 18 hours a day over a three-day weekend (if you're spending 18 hours a day training for a marathon, you'll probably end up in the hospital).

The analogy to learning a musical instrument is a particularly good one. There are certain facts and definitions in Econ 103 that you'll need to memorize, just as you need to memorize basic musical notation and the locations of the notes on the keyboard if you want to learn the piano. \emph{Merely} memorizing the basics, however, will no more ensure your success in this course than it will make you a concert pianist. Only regular practice \emph{applying} what you know can do that. If you wanted to learn the piano, I'd advise you to spend an hour each day. It's not essential that you study for Econ 103 every day, but you should expect to spend at least seven hours each week studying for this course, not counting time spent in lectures and recitations. If you don't put in the hours, you can't expect to do well.

\textbf{Please don't take this tip as a challenge.} There are neither bonus points nor glory for telling me after class that you only started studying for the exam at 4am that morning. Honestly, this would either serve as a horrible excuse for sub-par performance in the course, or leave me wondering how much better you could have done if you had spent more time learning the material. Both impressions reflect poorly on you. 

\section{Think; don't try to learn by rote}
Honestly, there are two ways to pass this course: the easy way and the hard way. The easy way is understanding the concepts. The hard way is trying to brute force the course. Unfortunately, most students will choose the hard way because it seems easier at the time. 

Think about learning addition. One approach would be to memorize every combination of numbers and their resulting sum. Hence, knowing the sums of any two numbers between 1 and 10 would simply be a task of memorizing 100 combinations. If you were particularly clever, you would take advantage of the commutativity of addition (e.g. 1 + 2 = 2 + 1) and only memorize 55 combinations. It's feasible, but anyone's brain would begin to melt once the teacher said you were responsible for knowing your sums of all numbers between 1 and 100 or 1 and 1000. If the teacher began testing on sums of 3 numbers, it would be game over. If you decided to take the easier route (or arguably, lazier, since it requires less work in the long run), you would begin understanding the actual \emph{concept} of addition. Once you understood that addition is simply incrementing one number by another number, then your addition abilities would only be limited by how high you could count. Taking an exam on sums of numbers between 1 and 10 would be no different from an exam on sums of numbers between 1 and 100 or even sums of 3, 4, or 10 different numbers. Econ 103 is the same. Small tweaks can exponentially increase difficulty for brute force methods, but are easily handled if you understand the underlying concepts.

Econ 103 will furnish you with many important skills and the homework will give you an opportunity to practice them. To pass the exams, you'll most certainly need to know these skills. To do \emph{well} on the exams, however, you'll need to demonstrate that you've learned how to \emph{think} statistically. \emph{Understanding} uncertainty is hard enough, let alone trying to \emph{quantify} it. In this course we'll do both. This won't be easy, but I hope to convince you over the course of the semester that it's worth the effort. For now, take my word for it. When you're working on the homework problems and R tutorials, get in the habit of asking yourself how this material relates to what you've learned to far, and focus on \emph{concepts} as well as skills.

\section{Go Through the Slides Carefully After Each Lecture}
While attending lectures is important, it's not enough. The only way to make sure that you don't fall behind in this course is to go through all the slides carefully after each lecture. (I'll post these on the course website.) This course is incredibly cumulative: ideas that we cover in the first few weeks of the semester re-appear again and again. If you don't come to a lecture with a good understanding of what we've covered so far, you'll only compound the problem, making it even harder to get caught up. The corollary to this is that if you understand the foundational concepts, each new concept will tacitly review the previous ones so when it comes time to study for the cumulative final, you will find that you remember the first lectures very well since we have been building upon them all semester. \textbf{This will save you a lot of study time and reduce your stress at the end of the semester.}

So how should you go about reviewing the slides? The first step is to make sure you understand all the \emph{basic definitions} and \emph{terminology} introduced in the lecture. This may seem obvious, but it's crucial: unless you first learn the \emph{language} of probability and statistics, you'll never be able to understand the \emph{ideas}. Once you have a good handle on the definitions, the next step is to make a list of the key \emph{concepts} introduced in the lecture. You may not completely understand them at this point, but that's ok. The point is to make sure you can articulate what it is that I'm trying to get you to understand in a given lecture. 

Once you know the definitions and have identified the key concepts (i.e. have a good overview of the topic), go back to the beginning of the lecture and work through everything line-by-line. If you find something that you don't understand, spend a few minutes thinking carefully about it. If you still don't understand after ten minutes, circle the section that's confused you and move on. When you reach the end of the slides, make a list of all the things you circled, with a brief description of what you're confused about. Expect to be confused by this material at first: it's hard! If you don't have any questions after working through a lecture, chances are you're not really grappling with the material.

Now you're in a great position. You have two lists: the first tells you exactly what the \emph{point} of this lecture was, and the second one tells you exactly what you don't understand. The first list will help you to focus on the ``big picture'' and the second one will help you to fill in the gaps in your knowledge. To get your questions answered, the best place to go is Piazza. Follow the checklist for using Piazza:
\begin{itemize}[nolistsep]
	\item \textbf{Has your question already been asked?} Browse through the questions that have already been posted. Chances are someone else has already answered your question.
	\item \textbf{Is there a related question to yours?} If there's a similar question that doesn't quite explain what you need to know, add a follow-up questions instead of creating a completely new question. This keeps related information in the same place, makes it easier to find answers on Piazza, and shows me that more than one person is confused about a certain issue, so I know it's worth spending time on.
	\item \textbf{If there's no post that answers your question and no post similar enough for you to post a follow-up, create a new question.} To increase the chance of getting a good answer \emph{be as clear and detailed as possible}. Make sure to explain exactly which lecture and slide you're confused about, otherwise we may not know what you're referring to. ``I don't understand statistical power'' is an example of a \emph{bad question} -- as written it doesn't have an answer. To turn this into a good question that is easy to answer, all you need to do is be clear about what you \emph{do} and \emph{do not} understand about statistical power. 
\end{itemize} 
Office hours and recitations are also a good place to ask questions. The main advantage of Piazza, however, is that it creates a record of every answer. This is a ``positive externality'' and something that I definitely want to encourage.

\section{Quizzes Cover the Basics to Keep You On Track}
There will be a number of quizzes over the course of the semester. (See the syllabus and course website for details.) These will be \emph{short} and \emph{easy}. The idea is to make sure you understand the \emph{basic material} from the lectures covered since the last quiz or exam -- definitions, formulas, and simple examples -- and reward you for staying on track. In particular, if you spend a couple of hours reviewing the slides you should expect to be able to get at least a ``B'' and very possibly an ``A'' on that week's quiz. Because they are intended to be easy, quizzes are \emph{not} a good guide to whether you are well prepared for an exam. If you consistently do well on the quizzes, you definitely know the basics. Exams, however, will go \emph{beyond} the basics. If you want to do well on them, the best measure of your understanding is the homework and past exams.

\section{Do the homework}
Homework will consist of problem sets and R Tutorials which I'll post on the course website. Problems sets are a combination of odd-numbered problems from the book and more challenging ``Additional Problems'' drawn from past exams and other sources. There are generally two approaches to using the homework. 

One way is to use it as a ``learn-by-doing'' tool where you go through the homework and use it as a guide to understanding the material. \textbf{If you do this, be sure to not confuse completing the homework with completely understanding the material.} I would encourage you to go through the homework a second time later without the use of notes or slides to ensure you have internalized the concepts. 

The second way is to use it to check your understanding. After you've gone through your notes and slides, work through the homework to see how much you understand, make note of the areas that you have to go back, and reference your notes for so you can strengthen your understanding of those concepts before the exam.

Much of the learning you'll do in this class takes place at home, not in the lectures. Homework is a crucial part of this. Working through difficult questions both deepens your understanding and points out gaps in your knowledge that you might otherwise have missed. This is why it is \emph{absolutely essential} that you make a serious attempt to solve a problem before looking at the solution.
If you're stuck, take a break and come back to it. Try to get a hint from Piazza or a classmate.
The solutions are there for you to \emph{check your work} and as a \emph{last resort} if you're really stuck. Don't fall into the trap of thinking that you can simply read through the solutions rather than working the problems yourself. If you do, you'll be in for an unpleasant surprise when midterms roll around.

\section{Yes, you really need to learn R}
R is an integral part of this course. To my mind there's no point teaching the theory of statistics without also giving students the tools to \emph{use} statistics. You'll learn to use R in a series of tutorials assigned as homework. Later in the semester, R material will be integrated directly into homework problems, enabling us to explore more interesting questions than you can handle with a pencil and paper.  Please be aware that I \emph{will} ask questions about R on exams. It's fairly clear, however, that a great deal of what's included in the R assignments can't easily be tested in-class. It would be a mistake, however, to conclude that you don't need to really \emph{do} the R assignments. 

First, it's quite easy to write questions that distinguish between who actually did the assignments versus who merely \emph{read} them. Second, and more importantly, the R material is carefully designed to strengthen your understanding of the concepts covered in class. The very best way to ensure that you understand a statistical procedure is to write computer code to implement it. I can't tell you how many times I've found myself flummoxed when trying to code up a procedure I thought I knew inside-out. Invariably, getting it to work on the computer has led me to a much deeper understanding of the theory. The R material can be tricky, so you should expect to be confused at first. Piazza is a great place to get hints and pointers if you find yourself stuck. Just be careful not to fall into the trap of merely reading someone else's solution and assuming you understand the assignment.

\end{document}
