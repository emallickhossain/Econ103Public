%\documentclass[handout]{beamer}
\documentclass{beamer}
 
\usetheme[numbering = fraction, progressbar = none, background = light, sectionpage = progressbar]{metropolis}
\usepackage{amsmath}
\usepackage{synttree}
\usepackage{tabu}
\usepackage{graphicx}

\title{Lecture 1 Review Quiz}
\author{Mallick Hossain}
\date{}
\institute{University of Pennsylvania}

\begin{document}

%%%%%%%%%%%%%%%%%%%%%%%%%%%%%%%%%%%%%%%%
\begin{frame}
	\titlepage 
\end{frame} 

%%%%%%%%%%%%%%%%%%%%%%%%%%%%%%%%%%%%%%%%
\begin{frame}
\frametitle{Questions About Class Logistics}
    \begin{enumerate}[<+- | alert@+>]
        \item Have you logged into Piazza?
            \begin{itemize}
                \item The correct answer is ``Yes''
            \end{itemize}
        \item Have you chosen a grading scheme?
            \begin{itemize}
                \item If you answered ``Yes'' to (1), then you've probably already answered this one too
            \end{itemize}
    \end{enumerate}
\end{frame} 

%%%%%%%%%%%%%%%%%%%%%%%%%%%%%%%%%%%%%%%%
\begin{frame}
\frametitle{Real Questions}
    \begin{enumerate}[<+- | alert@+>]
        \item What's the difference between a parameter and a statistic?
            \begin{itemize}
                \item A parameter is a characteristic of a \textit{population} and a statistic is a characteristic                      
                of a \textit{sample}.
            \end{itemize}
        \item Describe the two kinds of errors that can affect statistics. 
            \begin{itemize}
                \item \textit{Sampling} errors are random differences between the sample and the 
                population, but they should cancel out as the sample size grows
                \item \textit{Nonsampling} errors are systematic differences between the sample and the 
                population and will not cancel out as the sample grows larger
            \end{itemize}
        \item What is a confounder?
        \begin{itemize}
                \item A factor that influences both the treatment and the outcome. It masks the true 
                effect of the treatment
        \end{itemize}
        \item What's the best way to show causation?
        \begin{itemize}
            \item Double-blind randomized control trial
        \end{itemize}
    \end{enumerate}
\end{frame} 

%%%%%%%%%%%%%%%%%%%%%%%%%%%%%%%%%%%%%%%%
\begin{frame}
\frametitle{Real Questions}
    \begin{enumerate}[<+- | alert@+>]
    \setcounter{enumi}{4}
        \item What are the different types of variables?
            \begin{itemize}
                \item \textit{Discrete} variables can take on a countable number of values
                \item \textit{Continuous} variables can take any value
                \item \textit{Nominal} variables are simply categories
                \item \textit{Ordinal} variables have a natural ordering
                \item \textit{Interval} variables have meaningful differences, but do not have a natural zero
                \item \textit{Ratio} variables have meaningful differences and ratios, have a natural zero
            \end{itemize} 
            \item What is the formula for percentile?
                \begin{itemize}
                    \item $\frac{P}{100} * (n + 1) \text{position}$
                \end{itemize}
            \item Ready for the next lecture?
                \begin{itemize}
                    \item The correct answer is ``Yes''
                \end{itemize}
    \end{enumerate}
\end{frame} 

%%%%%%%%%%%%%%%%%%%%%%%%%%%%%%%%%%%%%%%%
\begin{frame}
\frametitle{Real Questions}
    \centering
    \includegraphics[scale=0.75]{letsDoThis1.jpg}
\end{frame} 

\end{document}