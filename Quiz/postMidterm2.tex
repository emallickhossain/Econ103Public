\begin{frame}{Test Statistics}

\begin{itemize}
\tightlist
\item
  Mean: 75
\item
  Median: 79
\item
  Standard Deviation: 20
\end{itemize}

\end{frame}

\begin{frame}{Common Mistakes}

\begin{enumerate}[<+->]
\def\labelenumi{\arabic{enumi}.}
\tightlist
\item
  Being careless with sums:

  \begin{itemize}[<+->]
  \tightlist
  \item
    \(\sum_i X_i \neq nX_i\)
  \item
    What is \(i\)?
  \end{itemize}
\end{enumerate}

\begin{enumerate}[<+->]
\def\labelenumi{\arabic{enumi}.}
\setcounter{enumi}{1}
\tightlist
\item
  \(\bar{X} \neq \mu\) anymore

  \begin{itemize}[<+->]
  \tightlist
  \item
    \(\mu\) is the true population mean. At the start of the course when
    we were dealing with \textbf{finite} samples, this was true.
  \item
    Now we are thinking of populations as \textbf{distributions} and
    samples are draws from that population. As a result, populations are
    conceptually \textbf{infinite}.
  \end{itemize}
\end{enumerate}

\end{frame}

\begin{frame}{Common Mistakes}

\begin{enumerate}[<+->]
\def\labelenumi{\arabic{enumi}.}
\setcounter{enumi}{2}
\tightlist
\item
  Be precise with your definitions and read over them to make sure they
  make sense. These are \textbf{not} definitions:

  \begin{itemize}[<+->]
  \tightlist
  \item
    ``A random variable is neither random nor a variable''
  \item
    ``A value that can take on any value of its realization''
  \item
    ``A variable\ldots{}and the specific value it takes is random''
  \end{itemize}
\end{enumerate}

\begin{enumerate}[<+->]
\def\labelenumi{\arabic{enumi}.}
\setcounter{enumi}{3}
\tightlist
\item
  When proving something, make sure you are not assuming what I asked
  you to prove:

  \begin{itemize}[<+->]
  \tightlist
  \item
    Prove that \(E[aX + b] = aE[X] + b\) is asking you to prove
    linearity.
  \end{itemize}
\end{enumerate}

\end{frame}

\begin{frame}{Common Mistakes}

\begin{enumerate}[<+->]
\def\labelenumi{\arabic{enumi}.}
\setcounter{enumi}{4}
\tightlist
\item
  \(\sum_i p(x) = 1\) is a property of pmfs. This has nothing to do with
  independence.
\end{enumerate}

\begin{enumerate}[<+->]
\def\labelenumi{\arabic{enumi}.}
\setcounter{enumi}{5}
\tightlist
\item
  \(E[X^2] \neq E[f(x)^2]\)

  \begin{itemize}[<+->]
  \tightlist
  \item
    Expectation has a very specific definition
  \end{itemize}
\end{enumerate}

\begin{enumerate}[<+->]
\def\labelenumi{\arabic{enumi}.}
\setcounter{enumi}{6}
\tightlist
\item
  Is \(Var(\sum_i X_i) = \sum_i Var(X_i)\) always true?
\end{enumerate}

\end{frame}

\begin{frame}{Next Steps}

I will provide the following opportunity to make up some points:

\begin{enumerate}[<+->]
\def\labelenumi{\arabic{enumi}.}
\tightlist
\item
  Pick \textbf{one} question from the midterm to redo
\item
  Rework it \textbf{by yourself}

  \begin{itemize}[<+->]
  \tightlist
  \item
    What does this mean?
  \item
    Many people will pick the same question
  \item
    I will NOT be happy or forgiving if I have any reason to question
    whether or not this is your work
  \end{itemize}
\item
  Write it up \textbf{neatly}. That means typed or VERY neatly written
\item
  Turn it in next week
\item
  You can earn up to half of the points you missed on that question
  back.
\item
  \textbf{This is part of the exam and everyone signed the Code of
  Academic Integrity statement. I expect everyone to uphold it.}
\end{enumerate}

\end{frame}
