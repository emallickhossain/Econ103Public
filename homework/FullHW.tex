\documentclass[addpoints,12pt]{exam}
\usepackage{amsmath, amssymb}
\usepackage{hyperref}
\linespread{1.1}

%To include the answers use:
%	pdflatex "\def\showanswers{1} \input{thisfile.tex}"
\ifdefined\showanswers
  \printanswers
\else
  \noprintanswers
\fi

\title{Problem Sets}
\author{Econ 103}
\date{}
\begin{document}
\maketitle

\section*{Part I -- Suggested Problems from the Textbook}
Chapter 1: 3, 5, 9, 11, 13
\\
Chapter 2: 1, 7, 8, 9bc, 13, 14, 16, 21, 23, 33, 35, 37, 41
\\
Chapter 3: 
\\
Chapter 4: 
\\
Chapter 5:
\\
Chapter 6:
\\
Chapter 7:
\\
Chapter 8:
\\
Chapter 9:
\\
Chapter 11:
\\
Chapter 12:
\\
Chapter 13:
\\
Chapter 14:

 \section*{Part II -- R Tutorial}
 \subsection*{Chapter 1 and Basics of R}
Complete R Tutorial 1 on the course website:
\\
\url{https://mallickhossain.wordpress.com/econ-103/}

This might seem like a long intro to R, but we need to get around the R learning curve quickly so you can get to the fun part, which is using R on real data! This should also serve as a useful reference for a majority of the tasks you will be expected to do in R.

\subsection*{Chapter 2}


\section*{Part III -- Additional Problems}
\begin{enumerate}
	\item You are a partner at Shady-Sleazy Consulting, LLC (motto: ``Everything you want to hear and nothing you don't!"\texttrademark). 
	\begin{enumerate}
		\item You have been hired by a large investment bank to help them convince their clients 			that they should sell Google stock. Create a chart for their slide deck that supports this 				view. Making it in Excel is fine, though I encourage you to try in R (using the ``quantmod''			package makes it easy to get stock data)
		\item You have been hired by the investment bank's rival as well and they want to convince 			their clients that they should invest more in a certain stock. As a partner of Shady-Sleazy 				Consulting, LLC, you have become adept at cutting corners, so you want to use the same 				data you've already collected. Create a chart that will convince their clients they should buy 			more Google stock.
	\end{enumerate}
\end{enumerate}

\end{document}