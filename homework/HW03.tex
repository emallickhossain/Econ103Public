\documentclass[addpoints,12pt]{exam}
\usepackage{amsmath, amssymb}
\linespread{1.1}
\usepackage{hyperref}
\usepackage{enumerate}
\usepackage[shortlabels]{enumitem}

%\printanswers

\title{Homework 3}
\author{Econ 103}
\date{}
\begin{document}
\maketitle

\section*{Lecture Progress}
We made it to slide 48 of the Chapter 3 lecture.

\section*{Homework Checklist}

\begin{itemize}[label = $\square$]
	\item \textbf{Book Problems (Chapter 3):} 1, 3, 5, 9, 11, 13, 15, 17ab, 35
	\item \textbf{Additional Problems: }See below
	\item\textbf{R Tutorial 3}
	\item \textbf{Ask questions on Piazza}
	\item\textbf{Review slides}
\end{itemize}

\section*{Additional Problems}
\begin{questions}

\question Suppose you flip a fair coin twice.
	\begin{parts}
		\part List all the basic outcomes in the sample space.
		\begin{solution}
			$S = \{HH, HT, TT, TH\}$
		\end{solution}
		\part Let $A$ be the event that you get at least one head. List all the basic outcomes in $A$.
		\begin{solution}
			$A = \{HH, HT, TH\}$
		\end{solution}
		\part What is the probability of $A$?
		\begin{solution}
			$P(A) = 3/4 = 0.75$
		\end{solution}
		\part List all the basic outcomes in $A^c$. 
		\begin{solution}
			$A^c = \{TT\}$
		\end{solution}
		\part What is the probability of $A^c$?
		\begin{solution}
			$P(A^c) = 1/4$
		\end{solution}
	\end{parts}

\question Suppose I deal two cards at random from a well-shuffled deck of 52 playing cards. What is the probability that I get a pair of aces? 
	\begin{solution}
	You can either solve this assuming that order doesn't matter:
		$$\frac{\binom{4}{2}}{\binom{52}{2}} = \frac{4!/(2!\times 2!)}{52!/(50!  \times 2!)} = \frac{6}{(52\times 51)/2}= 6/1326 = 1/221$$
		or that it does:
		$$\frac{P^4_2}{P^{52}_2} = \frac{4!/2!}{52!/50!} =\frac{(4\times 3)}{(52\times 51)} = 12/2652 = 1/221$$
		In either case, the answer is the same: $1/221  \approx 0.005$
	\end{solution}
	
\question In poker, a flush is when you have 5 cards of the same suit. What is the probability of getting a flush? For poker aficionados, we are including straight flushes and royal flushes, so don't exclude those.
	\begin{solution}
	By counting, there are 4 possible suits that you could get a flush with. Given a suit, there are 13 cards that we could hold and hence, there are $\dbinom{13}{5}$ possible combinations. This gives us
	$$
	\frac{\dbinom{4}{1} \dbinom{13}{5}}{\dbinom{52}{5}} = 0.00198
	$$
	\end{solution}

\question Suppose every Econ 103 student (assume there are 100 of you) set up a private trading firm. Every day, each student has a 50\% chance of beating the market. 
	\begin{parts}
		\part What is the probability that John Smith, a particular student in the class, beats the market five days in a row (exchanges are only open Monday-Friday)? 
			\begin{solution}
				$(1/2)^5 = 1/32\approx 0.03$
			\end{solution}
	 	\part What is the probability that at least one person beats the market five days in a row?
	 	\begin{solution}
	 	Use the complement rule: let $A$ be the event that at least one person gets five heads in a row. Calculate the probability that no one gets 5 heads in a row as follows:
	 		$$P(A^c) = (1 - 1/2^5)^{100} = (31/32)^{100}\approx 0.04$$
	 		Hence the desired probability is basically  $0.96$.
	 	\end{solution}
	 	\part What is the longest streak that someone in the class would be expected to pull off (with a greater than 50\% probability)? We assume that each student's outcome is independent.
	 	\begin{solution}
	 	The easiest way to do this is to generalize the formula we created in part (b). For a streak of length $n$, the probability that no one pulls it off is $(1 - \frac{1}{2}^n)^{100}$. If you plot this or just plug in a few longer streaks (6, 7, 8 days), you find that at 7 days, the chance of no on pulling of this streak is about 0.46 (at 8 days, it becomes about 0.68). Hence, the longest streak we could expect someone to pull off would be a seven day streak. 
	 	\end{solution}
	 	\part Based on the above, should we be surprised when some hedge funds pull off above-market annual returns (remember hedge funds don't normally tout their daily returns)? Why or why not? We'll assume hedge fund outcomes are independent of each other.
	 	\begin{solution}
	 	Using similar logic as above, but with the chances of annual returns instead of daily returns, we see that we should not be surprised that \textit{some} hedge fund performs better than the market for an extended period of time. However, we should be surprised if a \textit{particular} hedge fund performs better than the market for an extended period of time. 
	 	\end{solution}
	 	\part \textbf{Bonus (Hard):} Can you generalize the formula you found above to apply to any situation where we have $k$ participants, each with a probability $p$ of success and you can compute the probability of someone having a streak of length $n$?
	 	\begin{solution}
	 	Recall, the chance of someone having a streak of length $n$ is just the complement of no one having a streak of length $n$. Hence, we can write this as
	 	$$
	 	1 - (1 - p^n)^k
	 	$$
	 	where $p^n$ is the chance of getting a streak of length $n$. Therefore, $1 - p^n$ is the chance of not getting a streak of length $n$. Then, $(1 - p^n)^k$ is the chance that non of the $k$ participants get a streak of length $n$. The chance that someone gets this streak is simply the complement that no one gets the streak, hence $1 - (1 - p^n)^k$.
	 	\end{solution}
	\end{parts}

\question (Adapted from Mosteller, 1965) A jury has three members: the first flips a coin for each decision, and each of the remaining two independently has probability $p$ of reaching the correct decision. Call these two the ``serious'' jurors and the other the ``flippant'' juror (pun intended).
	\begin{parts}
		\part What is the probability that the serious jurors both reach the same decision?
			\begin{solution}
				There are two ways for them to agree: they can either make the right decision, $p^2$, or the wrong decision, $(1-p)^2$. These are mutually exclusive, so we sum the probabilities for a total of $p^2 + (1-p)^2$
			\end{solution}
		\part What is the probability that the serious jurors each reach different decisions?
			\begin{solution}
			There are two ways for them to disagree: either the first makes the wrong decision, $p(1-p)$, or the second makes the wrong decision, $(1-p)p$. These are mutually exclusive, so we sum the probabilities for a total of $2p(1-p)$.
			\end{solution}
	\part What is the probability that the jury reaches the correct decision? Majority rules.
	\begin{solution}
	 With probability $p^2$ the serious jurors agree and make the correct decision so the flippant juror is irrelevant. With probability $2p(1-p)$ they disagree. In half of these cases the flippant juror makes the correct decision. Thus, the overall probability is $p^2 + p(1-p) = p$.
	\end{solution}
	\end{parts}
	
\question Imagine that you're trying to find a suitable person to date. However, you're just too busy to spend all your time meeting people (and you prefer not to use Tinder). Hence, you've decided to save yourself some time and just hold interviews for those interested in being your boyfriend/girlfriend. On interview day, all of your suitors (suitresses?) gather in a room, hoping that they'll be the one you choose. You've made the following rules for the interviews:

\begin{itemize}
\item You don't know the suitors beforehand (you're too busy and you didn't peek in the room where they gathered)
\item You interview one person at a time (in a random order)
\item At the end of the interview, you either choose them or reject them 
\item If you reject them, you can't get them back
\item While your suitors could be ranked (there is a ``best'' suitor), you can only determine their relative rank (i.e. the second person you interviewed was better/smarter/hotter than the first).
\end{itemize} 

\begin{parts}
	\part Assume you have 3 suitors. If you've decided to accept the first person you interview, what's the probability you pick the best one?
	\begin{solution}
		The suitors enter in a random order, so we know there are 6 possible orders they could enter. 2 of these orderings have the best person entering first. Hence, you have a 1/3 chance of picking the best person. The orderings are 123, 132, 213, 231, 312, 321.
	\end{solution}
	\part What if you decide to pick the second person (no matter what)?
	\begin{solution}
		Just like in part (a), there are 6 possible permutations of suitors. 2 of them will have the best person entering second. You still have a 1/3 chance of picking the best suitor.
	\end{solution}
	\part You decide to be more sophisticated with your choosing since just picking someone based on the pure order they walk in seems like a suboptimal strategy. You refine your strategy as follows, you'll straight out reject the first person no matter what (poor guy/lady) and then pick the next person that is better than that (i.e. you'll pick person 2 only if he/she is better than person 1. Otherwise, you'll pick person 3). What is the chance of picking the best person using this strategy?
	\begin{solution}
		There are 6 possible orderings of suitors. In 2 of them (123 and 132), the best person enters first and you outright reject him (oh no!). What about the other 4 scenarios? If the second-best person is the first candidate (213 and 231), then you'll for sure pick the best person. Great! So in two scenarios, you'll for sure pick the best person. What about the remaining two scenarios (321 and 312)? In the first case, you'll pick the second-best person, but in the other, you'll pick the best person. Overall, you'll pick the best person in 3 out of 6 scenarios. Your probability increased to 50\%!
	\end{solution}
	\part Would your chances of picking the best suitor improve if you rejected the first 2 people instead?
	\begin{solution}
	No. The chance of getting the best suitor in the 3rd position is 1/3, so your chances have gone down. You're better off sticking with just rejecting the first person.
	\end{solution}
	\part What about if you had 4 suitors? How many should you outright reject to get the highest chance of finding the best one?
	\begin{solution}
	Since you have 4 suitors, there are 24 different orders that they could enter in. After listing the 24 possibilities, you should be able to come up with the following probabilities:
	\begin{center}
	\begin{tabular}{| c | c |}
	\hline
	\textbf{Number Rejected}		&		Probability of Best \\
	\hline
	1																&		11/24\\
	\hline
	2																& 	10/24\\
	\hline
	3																&		6/24\\
	\hline
	\end{tabular}
	\end{center}
	Rejecting the first one still gives you the best odds.
	\end{solution}
	\end{parts}

\section*{Part III -- Challenge Problem}
This is optional. If you're up for a challenge you'll learn quite a bit from this one.

\question Formally prove the Addition Rule: $P(A \cup B) = P(A) + P(B) - P(A \cap B)$ using the basic notions of set theory we learned in class and the axioms of probability. [Hint: Try to translate the intution from the Venn diagram into an equation.]
	\begin{solution}
		First, notice from the Venn diagram shown in class that we can partition $A\cup B$ into three mutually exclusive pieces: everything in $A$ that is \emph{not} in $B$, namely $A \cap B^c$,  everything in $B$ that is \emph{not} in $A$, namely $B \cap A^c$, and everything that is in \emph{both} $A$ and $B$, namely $A\cap B$. Writing this formally using our set theory notation:
		$$A \cup B = (A \cap B^c) \cup (B \cap A^c) \cup (A\cap B)$$
			where
			\begin{eqnarray*}
				(A \cap B^c) \cap (B \cap A^c)= \emptyset\\
				(A \cap B^c) \cap (A\cap B)= \emptyset\\
				(B \cap A^c) \cap (A\cap B) = \emptyset
			\end{eqnarray*}
		Now we can use two facts about probability. First, if two events are logically equivalent (contain exactly the same basic outcomes) then they have the same probability:
		$$P(A \cup B) = P\left[(A \cap B^c) \cup (B \cap A^c) \cup (A\cap B)\right]$$
Second, the probabilities of mutually exclusive events sum:
		$$P\left[(A \cap B^c) \cup (B \cap A^c) \cup (A\cap B)\right] = P(A \cap B^c) + P(B \cap A^c) + P(A\cap B)$$
Combining these two equations:
		$$P(A \cup B) = P(A \cap B^c) + P(B \cap A^c) + P(A\cap B)$$
		Now we're almost done. We want $P(A)$ rather than $P(A \cap B^c)$ and $P(B)$ rather than $P(B \cap A^c) $ on the right hand side of the equation. We also want a minus rather than a plus in front of $P(A\cap B)$. Just as we partitioned $A\cup B$ into mutually exclusive pieces, we can partition both $A$ and $B$ into mutually exclusive pieces as follows:
		\begin{eqnarray*}
			A &=& (A\cap B^c) \cup (A \cap B)\\
			B &=& (B\cap A^c) \cup (B \cap A)
		\end{eqnarray*}
And again using the fact that the probabilities of equivalent events are equal and the fact that the probabilities of mutually exclusive events sum,
		\begin{eqnarray*}
			P(A) &=& P(A\cap B^c) + P(A \cap B)\\
			P(B) &=& P(B\cap A^c) + P(A \cap B)
		\end{eqnarray*}
Rearranging, 
		\begin{eqnarray*}
			P(A\cap B^c)  &=&P(A) - P(A \cap B)\\
			P(B\cap A^c) &=& P(B)  - P(A \cap B)
		\end{eqnarray*}
Finally, substituting these two equations into our expression for $P(A\cup B)$ from above
	\begin{eqnarray*}
	P(A \cup B) &=& P(A \cap B^c) + P(B \cap A^c) + P(A\cap B)\\
	&=& \left[P(A) - P(A \cap B)\right] + \left[P(B)  - P(A \cap B)\right] + P(A\cap B)\\
	&=& P(A) + P(B) - P(A \cap B)
	\end{eqnarray*}
		\end{solution}

\end{questions}






\end{document}